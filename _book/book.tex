% Options for packages loaded elsewhere
\PassOptionsToPackage{unicode}{hyperref}
\PassOptionsToPackage{hyphens}{url}
%
\documentclass[
  a4paper,
]{article}
\usepackage[left=2cm,top=2cm,right=2cm]{geometry}
\title{Unser Kochbuch}
\author{Sabine und Balthasar}
\date{2023}

\usepackage{amsmath,amssymb}
% % \usepackage{lmodern}
% \usepackage{iftex}
\ifPDFTeX
  \usepackage[T1]{fontenc}
  \usepackage[utf8]{inputenc}
  \usepackage{textcomp} % provide euro and other symbols
\else % if luatex or xetex
  \usepackage{unicode-math}
  \usepackage{fontspec,xltxtra,xunicode}
  \defaultfontfeatures{Scale=MatchLowercase}
  % \defaultfontfeatures[\rmfamily]{Ligatures=TeX,Scale=1}
\fi
% Use upquote if available, for straight quotes in verbatim environments
\IfFileExists{upquote.sty}{\usepackage{upquote}}{}
\IfFileExists{microtype.sty}{% use microtype if available
  \usepackage[]{microtype}
  \UseMicrotypeSet[protrusion]{basicmath} % disable protrusion for tt fonts
}{}
\makeatletter
\@ifundefined{KOMAClassName}{% if non-KOMA class
  \IfFileExists{parskip.sty}{%
    \usepackage{parskip}
  }{% else
    \setlength{\parindent}{0pt}
    \setlength{\parskip}{6pt plus 2pt minus 1pt}}
}{% if KOMA class
  \KOMAoptions{parskip=half}}
\makeatother
\usepackage{xcolor}
\IfFileExists{xurl.sty}{\usepackage{xurl}}{} % add URL line breaks if available
\IfFileExists{bookmark.sty}{\usepackage{bookmark}}{\usepackage{hyperref}}
\hypersetup{
  pdftitle={Unser Kochbuch},
  pdfauthor={Sabine und Balthasar},
  hidelinks,
  pdfcreator={LaTeX via pandoc}}
\urlstyle{same} % disable monospaced font for URLs
\setlength{\emergencystretch}{3em} % prevent overfull lines
\providecommand{\tightlist}{%
  \setlength{\itemsep}{0pt}\setlength{\parskip}{0pt}}
\setcounter{secnumdepth}{-\maxdimen} % remove section numbering
\ifLuaTeX
  \usepackage{selnolig}  % disable illegal ligatures
\fi

\usepackage[nonumber]{cuisine}
\usepackage{xpatch}
\makeatletter
\xpatchcmd{\Displ@ySt@p}{\arabic{st@pnumber}}{}{}{}
\makeatother


% \usepackage{fontspec,xltxtra,xunicode}
%         \defaultfontfeatures{Mapping=tex-text}
%         \setmainfont[Mapping=tex-text]{Shobhika}

\usepackage{fontspec,xltxtra,xunicode}
    \defaultfontfeatures{Mapping=tex-text}
    % \setromanfont[Mapping=tex-text]{Helvetica} 
  % \setsansfont[Scale=MatchLowercase,Mapping=tex-text]{DejaVu Sans}
  % \setmathsfont(Latin,Digits){Junicode} 
  % \setmathsfont(Greek){Dialekt Uni} % or de-activate and use \pi etc. (looks better with \hat etc.)
  \newfontfamily\devanagarifont[Scale=0.8]{Devanagari MT}  


% \RecipeWidths{Total recipe width}{Step number width}{Number of servings width}    {Ingredient width}{Quantity width}{Units width}
% \RecipeWidths{.5\textwidth}{0cm}{1\textwidth}{3.5cm}{.5\textwidth}{1cm}

\renewcommand*{\recipetitlefont}{\Large\bfseries\sffamily}
\renewcommand*{\recipeservingsfont}{\itshape}
\renewcommand*{\recipetimefont}{\bfseries\itshape}
\renewcommand*{\recipefreeformfont}{\itshape}

\usepackage{titling}
\pretitle{\begin{center}\Huge\bfseries\sffamily}
 \posttitle{\par\end{center}\vskip 2cm}

 \preauthor{\begin{center}
            \Large\sffamily \lineskip 2cm%
            \begin{tabular}[t]{c}}
 \postauthor{\end{tabular}\par\end{center}}

\predate{\vfill\begin{center}\Large\sffamily}
 \postdate{\par\end{center}}

\begin{document}
\maketitle
\thispagestyle{empty}



\newpage
\vspace*{\fill}

\hfill\textit{Für Tara} \vspace*{\fill} \thispagestyle{empty} \newpage

\newpage
\vspace*{\fill}

\textbf{\Huge \sf Suppen} \vspace*{\fill} \thispagestyle{empty} \newpage

\begin{recipe}{Soupe à l’oignon}{Für ca. 4 Personen}{1h}

\ingredient{500 g}{Zwiebeln}
\ingredient{3 EL}{Butter}

\begin{quote}
Goldbraun anbraten, braucht ca. 40m, einkochen bis sie ganz weich sind.
\end{quote}

\freeform\hrulefill

\ingredient{3-4 L}{Geflügelfonds}
\ingredient{1 EL}{Balsamico-Essig}
\ingredient{}{Salz}
\ingredient{1-2}{Lorbeerblätter (fakultativ)}

\begin{quote}
Erhitzen und zu den Zwiebel giessen, alles gut aufkratzen; Essig dazu;
Salzen nach Geschmack (verträgt viel Salz!). Nach Belieben noch etwas
einkochen, oder zugedeckt die Lorbeerblätter ca. 10m mitkochen
\end{quote}

\freeform\hrulefill

\ingredient{4 Scheiben}{Weisses Brot}
\ingredient{150 g}{Gruyère oder Parmesan}

\begin{quote}
In Grösse der Suppenschalen schneiden, toasten. Dann Suppe in die
Schalen, Brot drauf, und schliesslich geriebener Käse drauf. Ca. 5m
unter dem Grill im Backofen (oberste Stufe) goldgelb überbacken und
sofort servieren.
\end{quote}

\freeform\hrulefill\newline\freeform{Wir machen diese Suppe immer aus einem Fonds, den wir durch stundenlanges (ca. 6h) Auskochen der Knochen von grilliertem Geflügel gewinne (z.B. Rumpf einer am Spiess gebratenen Weihnachtsgans.)}\end{recipe}

\newpage
\vspace*{\fill}

\textbf{\Huge \sf Gemüse} \vspace*{\fill} \thispagestyle{empty} \newpage

\begin{recipe}{Rotkraut}{Für ca. 4 Personen}{45m mit Dampfkochtopf; 2h sonst}

\ingredient{2-3}{Zwiebeln}
\ingredient{1}{Pepperoncino}
\ingredient{1 EL}{Butter}

\begin{quote}
Die Zwiebeln und den Pepperoncino fein schneiden und in Butter ca. 10m
dünsten bis sie leicht goldig sind.
\end{quote}

\freeform\hrulefill

\ingredient{1-2}{Rotkrautköpfe, je nach Grösse}

\begin{quote}
In feine Streifen schneiden und mitdünsten, ca. 10min
\end{quote}

\freeform\hrulefill

\ingredient{3 dl}{Rotwein}
\ingredient{}{Wasser}
\ingredient{}{Salz}

\begin{quote}
Mit dem Wein ablöschen, salzen nach Bedarf. Mit Wasser aufgiessen, bis
die Flüssigkeit sichtbar ist, aber nicht das Rotkraut übersteigt. 10min
im Dampfkochtopf auf 2. Stufe, anschliessend mit offenem Deckel so lange
einkochen, bis nur noch wenig Flüssigkeit bleibt, mindestens etwa 10min.
\end{quote}

\freeform\hrulefill\newline\freeform{}\end{recipe}

\newpage

\begin{recipe}{Randen}{Für ca. 4 Personen}{60m}

\ingredient{2-3}{Zwiebeln}
\ingredient{1}{Pepperoncino}
\ingredient{3-4}{Randen}
\ingredient{3}{Karotten}
\ingredient{}{nach Belieben anderes Wurzelgemüse, z.B. Pastinaken}
\ingredient{1 EL}{Butter}

\begin{quote}
Zwiebel und Pepperoncino hacken, alles andere in ca. 2cm grosse Stücke
schneiden und auf hoher Flamme mit genug Butter anbraten, dann auf
kleiner Flamme und geschlossenem Deckel fertigdünsten.
\end{quote}

\freeform\hrulefill\newline\freeform{}\end{recipe}

\newpage

\begin{recipe}{Dörrbohnen (Tüüri Boone)}{Für ca. 4 Personen}{30m mit Dampfkochtopf}

\ingredient{100 g}{Dörbohnen}

\begin{quote}
45-60m in kalten Wasser einweichen
\end{quote}

\freeform\hrulefill

\ingredient{5-6}{Gedörrte Tomaten oder frische Datteltomaten}
\ingredient{2 EL}{Butter}
\ingredient{1}{Pepperoncino (optional)}
\ingredient{2}{Zwiebeln}
\ingredient{1-2}{Knoblauchzehen}

\begin{quote}
Die Zwiebeln und den Pepperoncino fein schneiden und in Butter ca. 10m
dünsten bis sie leicht goldig sind. Falls gedörrte Tomaten verwendet
werden, diese auch in Stücke schneiden und von Anfang an mitbraten;
frische Datteltomaten erst am Schluss halbiert dazugeben. Auch gegen
Schluss den Knoblauch reinpressen. Anschliessend die abgetropften
Dörrbohnen dazugeben. (Einweichwasser weggiessen.) Im Dampfkochtopf mit
Wasser bis kurz unter die Bohnen ca. 10m kochen, anschliessend bei
offenem Deckel noch ca 20 einkochen.
\end{quote}

\freeform\hrulefill\newline\freeform{}\end{recipe}

\newpage

\begin{recipe}{Kartoffelgratin}{Für ca. 4 Personen}{60m}

\ingredient{2 kg}{Mehligkochende Kartoffeln}
\ingredient{1 EL}{Butter}
\ingredient{}{Salz}

\begin{quote}
Schälen und in dünne Scheiben schneiden, in eine bebutterte Gratinform
legen; gut salzen (verträgt ca 1 TL pro Schicht Kartoffeln!)
\end{quote}

\freeform\hrulefill

\ingredient{300 g}{Gruyère}

\begin{quote}
darüber reiben (grob, gut verteilt)
\end{quote}

\freeform\hrulefill

\ingredient{2 dl}{Halbrahm}
\ingredient{3 dl}{Weisswein}
\ingredient{}{Muskatnuss}
\ingredient{}{Pfeffer, frisch gemahlen}

\begin{quote}
vermischen und drüber giessen; es sollte soviel Flüssigkeit da sein,
dass sie bei leichtem Ankippen sichtbar ist. Anschliessend ca. 40 min im
Ofen bei ca. 180° backen bzw. bis die Kartoffeln gar sind (probieren!);
ggf Temperatur etwas runterstellen und länger backen.
\end{quote}

\freeform\hrulefill\newline\freeform{}\end{recipe}

\newpage

\begin{recipe}{Schwarzwurzeln}{Für ca. 4 Personen}{60m}

\ingredient{}{Schwarzwurzeln}
\ingredient{1 EL}{Butter}
\ingredient{}{Salz}

\begin{quote}
Schälen und in ca 2-3cm grosse Stücke schneiden (einmal längs); im
Butter anbraten und dann ca. 30 min auf kleiner Flamme schmoren; salzen
\end{quote}

\freeform\hrulefill

\ingredient{1 Bund}{Petersilie}

\begin{quote}
Hacken und kurz mitdünsten
\end{quote}

\freeform\hrulefill\newline\freeform{}\end{recipe}

\newpage

\begin{recipe}{आलू गोभी}{Für ca. 4 Personen}{60m}

\ingredient{2-4}{Blumenkohl}

\begin{quote}
Blumenkohl in Stücken auf einem Blech ca eine halbe Stunde grillieren.
\end{quote}

\freeform\hrulefill

\ingredient{2-3}{Zwiebeln}
\ingredient{1}{Pepperoncino}
\ingredient{1 TL}{Kreuzkümmel}
\ingredient{1 TL}{Fenchelsamen}
\ingredient{1 TL}{Kurkumapulver}
\ingredient{1 EL}{Ausgelassene Butter (घी)}

\begin{quote}
Alles zusammen in ausgelassener Butter anbraten; falls es anhockt, etwas
Wasser oder 3-4 Tomatenstücke. Sobald die Kartoffeln gar sind, die
grillierten Blumenkohlstücke dazu; evtl. mit etwas Wasser noch etwas
nachgaren
\end{quote}

\freeform\hrulefill\newline\freeform{}\end{recipe}

\newpage

\begin{recipe}{दाल}{Für ca. 4 Personen}{30m}

\ingredient{200 g}{Rote Linsen}
\ingredient{2 TL}{Turmeric}
\ingredient{2 TL}{Salz (nach Geschmack)}

\begin{quote}
In ca. 1-1.5l Wasser kochen bis zur gewünschten Konsistenz
\end{quote}

\freeform\hrulefill

\ingredient{2}{Zwiebeln}
\ingredient{1.5 - 2 cm}{Ingwer}
\ingredient{1 TL}{Kreuzkümmel}
\ingredient{2-3}{Curryblätter (wenn verfügbar)}
\ingredient{1 EL}{Butter}

\begin{quote}
Zwiebel und Ingwer sehr fein schneiden und in Butter dunkel anbraten.
Curryblätter kurz mit anbraten. Anschliessend zu den fertiggekochten
Linsen dazumischen.
\end{quote}

\freeform\hrulefill\newline\freeform{Nach einem Rezept von unserem Kollegen Yogendra Yādāva, mit dem Balthasar in den 90er Jahren über Maithili gearbeitet hat.}\end{recipe}

\newpage

\begin{recipe}{दाल}{Für ca. 4 Personen}{30m}

\ingredient{200 g}{Rote Linsen}
\ingredient{2 TL}{Turmeric}
\ingredient{2 TL}{Salz (nach Geschmack)}

\begin{quote}
In ca. 1-1.5l Wasser kochen bis zur gewünschten Konsistenz
\end{quote}

\freeform\hrulefill

\ingredient{2}{Zwiebeln}
\ingredient{1.5 - 2 cm}{Ingwer}
\ingredient{1 TL}{Kreuzkümmel}
\ingredient{2-3}{Curryblätter (wenn verfügbar)}
\ingredient{1 EL}{Butter}

\begin{quote}
Zwiebel und Ingwer sehr fein schneiden und in Butter dunkel anbraten.
Curryblätter kurz mit anbraten. Anschliessend zu den fertiggekochten
Linsen dazumischen.
\end{quote}

\freeform\hrulefill\newline\freeform{Nach einem Rezept von unserem Kollegen Yogendra Yādāva, mit dem Balthasar in den 90er Jahren über Maithili gearbeitet hat.}\end{recipe}

\newpage
\vspace*{\fill}

\textbf{\Huge \sf Geflügel und Eiergerichte} \vspace*{\fill}
\thispagestyle{empty} \newpage

\begin{recipe}{Coq au vin à l’italienne}{Für ca. 4 Personen}{60m}

\ingredient{8}{Pouletstücke}
\ingredient{}{Olivenöl}

\begin{quote}
Die Pouletstücke in Olivenöl anbraten bis sie goldbraun sind.
\end{quote}

\freeform\hrulefill

\ingredient{1 H}{Datteltomaten}
\ingredient{1 H}{Oliven (entsteint)}
\ingredient{3}{Knoblauchzehen}
\ingredient{1}{Zitrone (Schale davon)}
\ingredient{}{Pfeffer}
\ingredient{}{Salz, grobkörnig}

\begin{quote}
Die halbierten Tomaten dazugeben, unter die Pouletstücke schieben, kurz
mit anbraten (Flamme etwas reduzieren); dann Zitronenschale abreiben und
zusammen mit Knoblauch und Oliven klein hacken, auch dazugeben; salzen
und pfeffern
\end{quote}

\freeform\hrulefill

\ingredient{3dl}{Weisswein}

\begin{quote}
Mit dem Wein ablöschen, ca. 40min auf kleiner Flamme zugedeckt kochen.
\end{quote}

\freeform\hrulefill\newline\freeform{}\end{recipe}

\newpage

\begin{recipe}{एक्री}{Für ca. 4 Personen}{60m}

\ingredient{5 TL}{Korianderpulver}
\ingredient{2 TL}{Cayennepfeffer oder scharfer Paprika}
\ingredient{2 TL}{Fenchelsamen}
\ingredient{2 TL}{Kreuzkümmelpulver}
\ingredient{1.5 TL}{Turmeric}
\ingredient{1.5 cm}{Ingwer}
\ingredient{2 Zehen}{Knoblauch}

\begin{quote}
Im Mörser zu einer gleichmässigen Paste verarbeiten
\end{quote}

\freeform\hrulefill

\ingredient{4-5}{Zwiebeln}
\ingredient{1 TL}{Fenchelsamen}
\ingredient{1 EL}{Ausgelassene Butter (घी)}
\ingredient{1 Büchse}{Tomaten (in Stücken) oder 2 frische Tomaten}
\ingredient{1 Büchse}{Kokosmilch}

\begin{quote}
Zuerst Fenchelsamen kurz in die heisse Butter, dann Zwiebel dazu und
goldbraun braten (ca 20m). Anschliessend die Gewürzpaste mit anbraten
und dann mit den Tomaten und der Kokosmilch ablöschen
\end{quote}

\freeform\hrulefill

\ingredient{8}{Eier}

\begin{quote}
Hartkochen, schälen und halbiert ins Curry geben.
\end{quote}

\freeform\hrulefill\newline\freeform{Eier-Curry, benannt nach der Aussprache des Kondukteurs im Gorakhpur-Delhi Express; inspiriert von Camellia Panjabi’s 50 Great Curries of India}\end{recipe}

\newpage 
\vspace*{\fill}

\textbf{\Huge \sf Fleisch} \vspace*{\fill} \thispagestyle{empty}
\newpage

\begin{recipe}{Roastbeef}{Für ca. 4 Personen}{4h}

\ingredient{1 Bund}{Thymian}
\ingredient{1}{Zitrone, Schale davon}
\ingredient{4-5 Blätter}{Salbei}
\ingredient{Etwas}{Rosmarin}
\ingredient{1 EL}{Dijon-Senf}
\ingredient{1-2 TL}{Grobes Salz}
\ingredient{2 Gutsch}{Olivenöl}

\begin{quote}
Alles feinhacken (Ausser Thymian sind alle Kräuter fakultativ) und zu
einer gut streichbaren Paste vermischen
\end{quote}

\freeform\hrulefill

\ingredient{1 kg}{Entrecôte am Stück}
\ingredient{}{Öl}

\begin{quote}
Rundherum scharf anbraten (in gusseisener Pfanne), bis goldbraun.
Anschliessend mit der Kräuterpaste bestreiche und in den Ofen. Bei
Niedertemperatur nach Angabe des Herstellers. (\emph{V-Zug Zartgaren:}
einstellen auf Symbole `Kabel drin', `Kochhut mit Sternen' und `Pfeil
rechts' und auf Kerntemperatur 52° C; Kann nachher in Alufolie kurz warm
gehalten werden.)
\end{quote}

\freeform\hrulefill\newline\freeform{}\end{recipe}

\newpage 
\vspace*{\fill}

\textbf{\Huge \sf Dessert} \vspace*{\fill} \thispagestyle{empty}
\newpage

\begin{recipe}{Quarktorte}{Für ca. 8 Personen}{2h}

\ingredient{70 g}{Zucker}
\ingredient{140 g}{Butter (weich)}
\ingredient{210 g}{Mehl}
\ingredient{1}{Vanillezuck}
\ingredient{1}{Ei}

\begin{quote}
Zusamenkneten und eine halbe Stunde kaltstellen. Dann damit den Boden
eine Springfrom bestreichen.
\end{quote}

\freeform\hrulefill

\ingredient{3}{Eigelb}
\ingredient{1 Tasse}{Zucker}

\begin{quote}
gut mixen.
\end{quote}

\freeform\hrulefill

\ingredient{3}{Eiweiss}

\begin{quote}
steifschlagen
\end{quote}

\freeform\hrulefill

\ingredient{750 g}{3/4-Fett Quark}
\ingredient{200 g}{Crème fraiche}
\ingredient{2 EL}{Mehl}

Vorsichtig mit Eigelbmasse und Eiweiss mischen und damit Springform
auffüllen. Bei 180° mit starker Unterhitze ca. 40' backen.

\freeform\hrulefill\newline\freeform{}\end{recipe}

\end{document}
