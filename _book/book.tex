% Options for packages loaded elsewhere
\PassOptionsToPackage{unicode}{hyperref}
\PassOptionsToPackage{hyphens}{url}
%
\documentclass[
  a4paper,
]{article}
\usepackage[left=2cm,top=2cm,right=2cm]{geometry}
\title{Unser Kochbuch}
\author{Sabine und Balthasar}
\date{2023}

\usepackage{amsmath,amssymb}
% % \usepackage{lmodern}
% \usepackage{iftex}
\ifPDFTeX
  \usepackage[T1]{fontenc}
  \usepackage[utf8]{inputenc}
  \usepackage{textcomp} % provide euro and other symbols
\else % if luatex or xetex
  \usepackage{unicode-math}
  \usepackage{fontspec,xltxtra,xunicode}
  \defaultfontfeatures{Scale=MatchLowercase}
  % \defaultfontfeatures[\rmfamily]{Ligatures=TeX,Scale=1}
\fi
% Use upquote if available, for straight quotes in verbatim environments
\IfFileExists{upquote.sty}{\usepackage{upquote}}{}
\IfFileExists{microtype.sty}{% use microtype if available
  \usepackage[]{microtype}
  \UseMicrotypeSet[protrusion]{basicmath} % disable protrusion for tt fonts
}{}
\makeatletter
\@ifundefined{KOMAClassName}{% if non-KOMA class
  \IfFileExists{parskip.sty}{%
    \usepackage{parskip}
  }{% else
    \setlength{\parindent}{0pt}
    \setlength{\parskip}{6pt plus 2pt minus 1pt}}
}{% if KOMA class
  \KOMAoptions{parskip=half}}
\makeatother
\usepackage{xcolor}
\IfFileExists{xurl.sty}{\usepackage{xurl}}{} % add URL line breaks if available
\IfFileExists{bookmark.sty}{\usepackage{bookmark}}{\usepackage{hyperref}}
\hypersetup{
  pdftitle={Unser Kochbuch},
  pdfauthor={Sabine und Balthasar},
  hidelinks,
  pdfcreator={LaTeX via pandoc}}
\urlstyle{same} % disable monospaced font for URLs
\setlength{\emergencystretch}{3em} % prevent overfull lines
\providecommand{\tightlist}{%
  \setlength{\itemsep}{0pt}\setlength{\parskip}{0pt}}
\setcounter{secnumdepth}{-\maxdimen} % remove section numbering
\ifLuaTeX
  \usepackage{selnolig}  % disable illegal ligatures
\fi

\usepackage[nonumber]{cuisine}
\usepackage{xpatch}
\makeatletter
\xpatchcmd{\Displ@ySt@p}{\arabic{st@pnumber}}{}{}{}
\makeatother


% \usepackage{fontspec,xltxtra,xunicode}
%         \defaultfontfeatures{Mapping=tex-text}
%         \setmainfont[Mapping=tex-text]{Shobhika}

\usepackage{fontspec,xltxtra,xunicode}
    \defaultfontfeatures{Mapping=tex-text}
    % \setromanfont[Mapping=tex-text]{Helvetica} 
  % \setsansfont[Scale=MatchLowercase,Mapping=tex-text]{DejaVu Sans}
  % \setmathsfont(Latin,Digits){Junicode} 
  % \setmathsfont(Greek){Dialekt Uni} % or de-activate and use \pi etc. (looks better with \hat etc.)
  \newfontfamily\Dev[Scale=0.8]{Devanagari MT}  


% \RecipeWidths{Total recipe width}{Step number width}{Number of servings width}    {Ingredient width}{Quantity width}{Units width}
% \RecipeWidths{.5\textwidth}{0cm}{1\textwidth}{3.5cm}{.5\textwidth}{1cm}

\renewcommand*{\recipetitlefont}{\Large\bfseries\sffamily}
\renewcommand*{\recipeservingsfont}{\itshape}
\renewcommand*{\recipetimefont}{\bfseries\itshape}
\renewcommand*{\recipefreeformfont}{\itshape}

\usepackage{titling}
\pretitle{\begin{center}\Huge\bfseries\sffamily}
 \posttitle{\par\end{center}\vskip 2cm}

 \preauthor{\begin{center}
            \Large\sffamily \lineskip 2cm%
            \begin{tabular}[t]{c}}
 \postauthor{\end{tabular}\par\end{center}}

\predate{\vfill\begin{center}\Large\sffamily}
 \postdate{\par\end{center}}

\begin{document}
\maketitle
\thispagestyle{empty}



\newpage
\vspace*{\fill}

\hfill\textit{\Large Für Tara} \vspace*{\fill} \thispagestyle{empty}
\newpage

\newpage
\vspace*{\fill}

\textbf{\Huge \sf Suppen} \vspace*{\fill} \thispagestyle{empty} \newpage

\begin{recipe}{Soupe à l’oignon}{Für ca. 4 Personen}{1h}

\ingredient{500 g}{Zwiebeln}
\ingredient{3 EL}{Butter}

\begin{quote}
Goldbraun anbraten, braucht ca. 40m, einkochen bis sie ganz weich sind.
\end{quote}

\freeform\hrulefill

\ingredient{3-4 L}{Geflügelfonds}
\ingredient{1 EL}{Balsamico-Essig}
\ingredient{}{Salz}
\ingredient{1-2}{Lorbeerblätter (fakultativ)}

\begin{quote}
Erhitzen und zu den Zwiebel giessen, alles gut aufkratzen; Essig dazu;
Salzen nach Geschmack (verträgt viel Salz!). Nach Belieben noch etwas
einkochen, oder zugedeckt die Lorbeerblätter ca. 10m mitkochen
\end{quote}

\freeform\hrulefill

\ingredient{4 Scheiben}{Weisses Brot}
\ingredient{150 g}{Gruyère oder Parmesan}

\begin{quote}
In Grösse der Suppenschalen schneiden, toasten. Dann Suppe in die
Schalen, Brot drauf, und schliesslich geriebener Käse drauf. Ca. 5m
unter dem Grill im Backofen (oberste Stufe) goldgelb überbacken und
sofort servieren.
\end{quote}

\freeform\hrulefill\newline\freeform{Wir machen diese Suppe immer aus einem Fonds, den wir durch stundenlanges (ca. 6h) Auskochen der Knochen von grilliertem Geflügel gewinne (z.B. Rumpf einer am Spiess gebratenen Weihnachtsgans.)}\end{recipe}

\newpage
\vspace*{\fill}

\textbf{\Huge \sf Pasta} \vspace*{\fill} \thispagestyle{empty} \newpage

\begin{recipe}{Älplermagronen}{Für ca. 6 Personen}{30m}

\ingredient{500 g}{Magronen}
\ingredient{800 g}{fest kochende Kartoffeln}
\ingredient{1 L}{Gemüsebouillon}
\ingredient{4 dl}{Halbrahm}

\begin{quote}
Magronen und in kleine Stücke geschnittene Kartoffeln in Bouillon und
Rahm ca. 12-15min kochen bis die Flüssigkeit absorbiert ist (rühren!)
\end{quote}

\freeform\hrulefill

\ingredient{4}{Zwiebeln}
\ingredient{4}{Knoblauchzehen}
\ingredient{2 EL}{Butter}
\ingredient{2 H}{Gruyère}

\begin{quote}
Braunbraten und mit geriebenem Gruyère unter die Magronen mischen
\end{quote}

\freeform\hrulefill\newline\freeform{}\end{recipe}

\newpage

\begin{recipe}{Broccoli-Pasta}{Für ca. 4 Personen}{30m}

\ingredient{6}{kleine Tomaten (oder weniger grosse)}
\ingredient{}{Salz}
\ingredient{}{Pfeffer}

\begin{quote}
Tomaten in Viertel schneiden, salzen und pfeffern; ca. 25' unter den
Grill.
\end{quote}

\freeform\hrulefill

\ingredient{1 kg}{Broccoli}
\ingredient{500 g}{Farfalle}

\begin{quote}
In Stücke schneiden und in einem grossen Topf kochendem Wasser kochen
bis sie al dente sind. Mit einem Sieblöffel rausnehmen und abschrecken.
Dann die Farfalle im Wasser al dente kochen. Derweil die Sauce
zubereiten.
\end{quote}

\freeform\hrulefill

\ingredient{4 EL}{Olivenöl}
\ingredient{4 EL}{Parmesan}
\ingredient{2}{Zitronen}
\ingredient{1 H}{Basilikum (o. Petersilie)}
\ingredient{}{Pfeffer}
\ingredient{}{Salz}

\begin{quote}
Alles gut vermischen und am Schluss mit Broccoli und Farfalle
vermischen.
\end{quote}

\freeform\hrulefill\newline\freeform{}\end{recipe}

\newpage
\vspace*{\fill}

\textbf{\Huge \sf Gemüse} \vspace*{\fill} \thispagestyle{empty} \newpage

\begin{recipe}{\Dev आलू गोभी}{Für ca. 4 Personen}{60m}

\ingredient{2-4}{Blumenkohl}

\begin{quote}
Blumenkohl in Stücken auf einem Blech ca eine halbe Stunde grillieren.
\end{quote}

\freeform\hrulefill

\ingredient{2-3}{Zwiebeln}
\ingredient{1}{Pepperoncino}
\ingredient{1 TL}{Kreuzkümmel}
\ingredient{1 TL}{Fenchelsamen}
\ingredient{1 TL}{Kurkumapulver}
\ingredient{1 EL}{Ausgelassene Butter ({\Dev घी})}

\begin{quote}
Alles zusammen in ausgelassener Butter anbraten; falls es anhockt, etwas
Wasser oder 3-4 Tomatenstücke. Sobald die Kartoffeln gar sind, die
grillierten Blumenkohlstücke dazu; evtl. mit etwas Wasser noch etwas
nachgaren
\end{quote}

\freeform\hrulefill\newline\freeform{}\end{recipe}

\newpage

\begin{recipe}{Artischoken}{Für ca. 4 Personen}{1h}

\ingredient{4}{Artischoken}
\ingredient{1}{Zitrone}

\begin{quote}
Die Spitzen der Blätter mit der Schere und Stil mit Messer abschneiden;
sofort mit Zitronensaft abreiben, damit nichts oxidiert. Die Artischoken
kopfüber für ca 1h in den Dampf geben (entweder auf kleinen Gestellen
über kochendem Wasser oder im Dämpfer)
\end{quote}

\freeform\hrulefill

\ingredient{500 g}{Quark}
\ingredient{2 EL}{Olivenöl}
\ingredient{2 Gabeln}{Kapern}
\ingredient{1 H}{Schnittlauch (H = ä Hampfle)}
\ingredient{1 H}{Petersilie}
\ingredient{1}{Schale einer Zitrone}
\ingredient{}{Salz}
\ingredient{1}{Knoblauchzehe}
\ingredient{1 TL}{Senf}

\begin{quote}
Sauce: Kräuter und Kapern fein hacken, Knoblauch pressen und mit allem
gut vermischen
\end{quote}

\freeform\hrulefill\newline\freeform{}\end{recipe}

\newpage

\begin{recipe}{\Dev दाल}{Für ca. 4 Personen}{30m}

\ingredient{200 g}{Rote Linsen}
\ingredient{2 TL}{Turmeric}
\ingredient{2 TL}{Salz (nach Geschmack)}

\begin{quote}
In ca. 1-1.5l Wasser kochen bis zur gewünschten Konsistenz
\end{quote}

\freeform\hrulefill

\ingredient{2}{Zwiebeln}
\ingredient{1.5 - 2 cm}{Ingwer}
\ingredient{1 TL}{Kreuzkümmel}
\ingredient{2-3}{Curryblätter (wenn verfügbar)}
\ingredient{1 EL}{Butter}

\begin{quote}
Zwiebel und Ingwer sehr fein schneiden und in Butter dunkel anbraten.
Curryblätter kurz mit anbraten. Anschliessend zu den fertiggekochten
Linsen dazumischen.
\end{quote}

\freeform\hrulefill\newline\freeform{Nach einem Rezept von unserem Kollegen Yogendra Yādāva, mit dem Balthasar in den 90er Jahren über Maithili gearbeitet hat.}\end{recipe}

\newpage

\begin{recipe}{Dörrbohnen (Tüüri Boone)}{Für ca. 4 Personen}{30m mit Dampfkochtopf}

\ingredient{100 g}{Dörbohnen}

\begin{quote}
45-60m in kalten Wasser einweichen
\end{quote}

\freeform\hrulefill

\ingredient{5-6}{Gedörrte Tomaten oder frische Datteltomaten}
\ingredient{2 EL}{Butter}
\ingredient{1}{Pepperoncino (optional)}
\ingredient{2}{Zwiebeln}
\ingredient{1-2}{Knoblauchzehen}

\begin{quote}
Die Zwiebeln und den Pepperoncino fein schneiden und in Butter ca. 10m
dünsten bis sie leicht goldig sind. Falls gedörrte Tomaten verwendet
werden, diese auch in Stücke schneiden und von Anfang an mitbraten;
frische Datteltomaten erst am Schluss halbiert dazugeben. Auch gegen
Schluss den Knoblauch reinpressen. Anschliessend die abgetropften
Dörrbohnen dazugeben. (Einweichwasser weggiessen.) Im Dampfkochtopf mit
Wasser bis kurz unter die Bohnen ca. 10m kochen, anschliessend bei
offenem Deckel noch ca 20 einkochen.
\end{quote}

\freeform\hrulefill\newline\freeform{}\end{recipe}

\newpage

\begin{recipe}{Kartoffelgratin}{Für ca. 4 Personen}{60m}

\ingredient{2 kg}{Mehligkochende Kartoffeln}
\ingredient{1 EL}{Butter}
\ingredient{}{Salz}

\begin{quote}
Schälen und in dünne Scheiben schneiden, in eine bebutterte Gratinform
legen; gut salzen (verträgt ca 1 TL pro Schicht Kartoffeln!)
\end{quote}

\freeform\hrulefill

\ingredient{300 g}{Gruyère}

\begin{quote}
darüber reiben (grob, gut verteilt)
\end{quote}

\freeform\hrulefill

\ingredient{2 dl}{Halbrahm}
\ingredient{3 dl}{Weisswein}
\ingredient{}{Muskatnuss}
\ingredient{}{Pfeffer, frisch gemahlen}

\begin{quote}
vermischen und drüber giessen; es sollte soviel Flüssigkeit da sein,
dass sie bei leichtem Ankippen sichtbar ist. Anschliessend ca. 40 min im
Ofen bei ca. 180° backen bzw. bis die Kartoffeln gar sind (probieren!);
ggf Temperatur etwas runterstellen und länger backen.
\end{quote}

\freeform\hrulefill\newline\freeform{}\end{recipe}

\newpage

\begin{recipe}{Randen}{Für ca. 4 Personen}{60m}

\ingredient{2-3}{Zwiebeln}
\ingredient{1}{Pepperoncino}
\ingredient{3-4}{Randen}
\ingredient{3}{Karotten}
\ingredient{}{nach Belieben anderes Wurzelgemüse, z.B. Pastinaken}
\ingredient{1 EL}{Butter}

\begin{quote}
Zwiebel und Pepperoncino hacken, alles andere in ca. 2cm grosse Stücke
schneiden und auf hoher Flamme mit genug Butter anbraten, dann auf
kleiner Flamme und geschlossenem Deckel fertigdünsten.
\end{quote}

\freeform\hrulefill\newline\freeform{}\end{recipe}

\newpage

\begin{recipe}{Randensteak}{Für ca. 4 Personen}{60m}

\ingredient{4-5}{Randen}
\ingredient{}{Olivenöl}

\begin{quote}
Randen in breite Scheiben schneiden, mit Öl bestreichen, salzen und
pfeffern. Ca. 45' unter dem Grill im Ofen auf Backpapier grillieren. Mit
Rotweinsauce servieren.
\end{quote}

\freeform\hrulefill\newline\freeform{}\end{recipe}

\newpage

\begin{recipe}{Rotkraut}{Für ca. 4 Personen}{45m mit Dampfkochtopf; 2h sonst}

\ingredient{2-3}{Zwiebeln}
\ingredient{1}{Pepperoncino}
\ingredient{1 EL}{Butter}

\begin{quote}
Die Zwiebeln und den Pepperoncino fein schneiden und in Butter ca. 10m
dünsten bis sie leicht goldig sind.
\end{quote}

\freeform\hrulefill

\ingredient{1-2}{Rotkrautköpfe, je nach Grösse}

\begin{quote}
In feine Streifen schneiden und mitdünsten, ca. 10min
\end{quote}

\freeform\hrulefill

\ingredient{3 dl}{Rotwein}
\ingredient{}{Wasser}
\ingredient{}{Salz}

\begin{quote}
Mit dem Wein ablöschen, salzen nach Bedarf. Mit Wasser aufgiessen, bis
die Flüssigkeit sichtbar ist, aber nicht das Rotkraut übersteigt. 10min
im Dampfkochtopf auf 2. Stufe, anschliessend mit offenem Deckel so lange
einkochen, bis nur noch wenig Flüssigkeit bleibt, mindestens etwa 10min.
\end{quote}

\freeform\hrulefill\newline\freeform{}\end{recipe}

\newpage

\begin{recipe}{Rotweinsauce}{Für ca. 4 Personen}{40m}

\ingredient{1}{Zwiebel}
\ingredient{1 H}{Pilze (optional)}
\ingredient{1 H}{Salbei (optional)}
\ingredient{1 EL}{Butter}
\ingredient{1 cm}{Tomatenmark (optional)}

\begin{quote}
Alles sehr fein hacken und in reichlich Butter ca. 30' dünsten. Die
Zwiebeln sollten weich sein.
\end{quote}

\freeform\hrulefill

\ingredient{2 EL}{Mehl}
\ingredient{.5 dl}{Rotwein}
\ingredient{.5 dl}{Bouillon (oder 1 dl Rotwein)}

\begin{quote}
Alles mit Mehl bestäuben und dann mit der Flüssigkeit ablöschen. Ca. 10'
weiterkochen, bis die Sauce dickflüssig ist (falls die Sauce zu dünn
ist, etwas in Wein aufgelöstes Maizena daruntermischen; falls sie zu
dick ist, mit etwas Wein strecken)
\end{quote}

\freeform\hrulefill\newline\freeform{}\end{recipe}

\newpage

\begin{recipe}{Schwarzwurzeln}{Für ca. 4 Personen}{60m}

\ingredient{}{Schwarzwurzeln}
\ingredient{1 EL}{Butter}
\ingredient{}{Salz}

\begin{quote}
Schälen und in ca 2-3cm grosse Stücke schneiden (einmal längs); im
Butter anbraten und dann ca. 30 min auf kleiner Flamme schmoren; salzen
\end{quote}

\freeform\hrulefill

\ingredient{1 Bund}{Petersilie}

\begin{quote}
Hacken und kurz mitdünsten
\end{quote}

\freeform\hrulefill\newline\freeform{}\end{recipe}

\newpage

\begin{recipe}{Weisser Spargel im Papier}{Für ca. 4 Personen}{30m}

\ingredient{1 kg}{Weisser Spargel}
\ingredient{3}{Zitronen}
\ingredient{2 EL}{Olivenöl}
\ingredient{}{Pfeffer}
\ingredient{}{grobes Salz}

\begin{quote}
Spargeln schälen und auf Backpapier legen. Zitronenschale, Olivenöl,
Pfeffer und Salz drauf verteilen und zu Portionen packen. ca. 1h bei
180° im Ofen
\end{quote}

\freeform\hrulefill\newline\freeform{}\end{recipe}

\newpage
\vspace*{\fill}

\textbf{\Huge \sf Geflügel und Eiergerichte} \vspace*{\fill}
\thispagestyle{empty} \newpage

\begin{recipe}{Coq au vin à l’italienne}{Für ca. 4 Personen}{60m}

\ingredient{8}{Pouletstücke}
\ingredient{}{Olivenöl}

\begin{quote}
Die Pouletstücke in Olivenöl anbraten bis sie goldbraun sind.
\end{quote}

\freeform\hrulefill

\ingredient{1 H}{Datteltomaten}
\ingredient{1 H}{Oliven (entsteint)}
\ingredient{3}{Knoblauchzehen}
\ingredient{1}{Zitrone (Schale davon)}
\ingredient{}{Pfeffer}
\ingredient{}{Salz, grobkörnig}

\begin{quote}
Die halbierten Tomaten dazugeben, unter die Pouletstücke schieben, kurz
mit anbraten (Flamme etwas reduzieren); dann Zitronenschale abreiben und
zusammen mit Knoblauch und Oliven klein hacken, auch dazugeben; salzen
und pfeffern
\end{quote}

\freeform\hrulefill

\ingredient{3dl}{Weisswein}

\begin{quote}
Mit dem Wein ablöschen, ca. 40min auf kleiner Flamme zugedeckt kochen.
\end{quote}

\freeform\hrulefill\newline\freeform{}\end{recipe}

\newpage

\begin{recipe}{\Dev एक्री}{Für ca. 4 Personen}{60m}

\ingredient{5 TL}{Korianderpulver}
\ingredient{2 TL}{Cayennepfeffer oder scharfer Paprika}
\ingredient{2 TL}{Fenchelsamen}
\ingredient{2 TL}{Kreuzkümmelpulver}
\ingredient{1.5 TL}{Turmeric}
\ingredient{1.5 cm}{Ingwer}
\ingredient{2 Zehen}{Knoblauch}

\begin{quote}
Im Mörser zu einer gleichmässigen Paste verarbeiten
\end{quote}

\freeform\hrulefill

\ingredient{4-5}{Zwiebeln}
\ingredient{1 TL}{Fenchelsamen}
\ingredient{1 EL}{Ausgelassene Butter ({\Dev घी})}
\ingredient{1 Büchse}{Tomaten (in Stücken) oder 2 frische Tomaten}
\ingredient{1 Büchse}{Kokosmilch}

\begin{quote}
Zuerst Fenchelsamen kurz in die heisse Butter, dann Zwiebel dazu und
goldbraun braten (ca 20m). Anschliessend die Gewürzpaste mit anbraten
und dann mit den Tomaten und der Kokosmilch ablöschen
\end{quote}

\freeform\hrulefill

\ingredient{8}{Eier}

\begin{quote}
Hartkochen, schälen und halbiert ins Curry geben.
\end{quote}

\freeform\hrulefill\newline\freeform{Eier-Curry, benannt nach der Aussprache des Kondukteurs im Gorakhpur-Delhi Express; inspiriert von Camellia Panjabi’s 50 Great Curries of India}\end{recipe}

\newpage 
\vspace*{\fill}

\textbf{\Huge \sf Fleisch} \vspace*{\fill} \thispagestyle{empty}
\newpage

\begin{recipe}{Fleischpflanzl}{Für ca. 4 Personen}{30m}

\ingredient{3}{Semmeln}

\begin{quote}
in warmen Wasser aufweichen. Ausdrücken.
\end{quote}

\freeform\hrulefill

\ingredient{2-3}{Zwiebeln}

\begin{quote}
Zwiebeln kleinschneiden.
\end{quote}

\freeform\hrulefill

\ingredient{1 kg}{Hackfleisch (halb/halb)}
\ingredient{1 TL}{Salz oder nach Bedarf}
\ingredient{etwas}{Majoran}
\ingredient{etwas}{Pfeffer}

\begin{quote}
Alles zusammenmischen. Handvoll der Masse flachdrücken und alles in
einer Pfanne anbraten.
\end{quote}

\freeform\hrulefill\newline\freeform{Omas Fleischpflanzl}\end{recipe}

\newpage

\begin{recipe}{Roastbeef}{Für ca. 4 Personen}{4h}

\ingredient{1 Bund}{Thymian}
\ingredient{1}{Zitrone, Schale davon}
\ingredient{4-5 Blätter}{Salbei}
\ingredient{Etwas}{Rosmarin}
\ingredient{1 EL}{Dijon-Senf}
\ingredient{1-2 TL}{Grobes Salz}
\ingredient{2 Gutsch}{Olivenöl}

\begin{quote}
Alles feinhacken (Ausser Thymian sind alle Kräuter fakultativ) und zu
einer gut streichbaren Paste vermischen
\end{quote}

\freeform\hrulefill

\ingredient{1 kg}{Entrecôte am Stück}
\ingredient{}{Öl}

\begin{quote}
Rundherum scharf anbraten (in gusseisener Pfanne), bis goldbraun.
Anschliessend mit der Kräuterpaste bestreiche und in den Ofen. Bei
Niedertemperatur nach Angabe des Herstellers. (\emph{V-Zug Zartgaren:}
einstellen auf Symbole `Kabel drin', `Kochhut mit Sternen' und `Pfeil
rechts' und auf Kerntemperatur 52° C; Kann nachher in Alufolie kurz warm
gehalten werden.)
\end{quote}

\freeform\hrulefill\newline\freeform{}\end{recipe}

\newpage

\vspace*{\fill}

\textbf{\Huge \sf Dessert} \vspace*{\fill} \thispagestyle{empty}
\newpage

\begin{recipe}{Ameisenkuchen}{Für ca. 4 Personen}{60m}

\ingredient{250 g}{Butter}

\begin{quote}
Butter schaumig rühren.
\end{quote}

\freeform\hrulefill

\ingredient{180 g}{Zucker}
\ingredient{1 Pckg}{Vanillezucker}
\ingredient{4}{Eier}
\ingredient{250 g}{Mehl}
\ingredient{1 Tasse}{Eierlikör}
\ingredient{1 Tasse}{Schoggistreusel}

\begin{quote}
Abwechslungsweise dazugeben. 50 Minuaten bei ca. 190 Grad backen.
\end{quote}

\freeform\hrulefill\newline\freeform{Omas Ameisenkuchen}\end{recipe}

\newpage

\begin{recipe}{Apfelkuchen 123}{Für ca. 10 Personen}{1 h}

\ingredient{70 g}{Zucker}
\ingredient{140 g}{Butter}
\ingredient{210 g}{Mehl}
\ingredient{1 Prise}{Salz}
\ingredient{1}{Ei}
\ingredient{1 Päckchen}{Vanillezucker}

\begin{quote}
Alle Zutaten zu einer Kugel kneten. Dann in eine runde Tarteform auf
Backpapierlegen. Mit Gabel einstechen. Etwas kaltstellen oder auch nicht
;-). Je nach Zeit, Lust und Laune.
\end{quote}

\freeform\hrulefill

\ingredient{x}{Äpfel}
\ingredient{etwas}{Zucker}

\begin{quote}
Ofen vorheizen auf 180 Grad. Ober und Unterhitze oder 200 Grad Umluft.
Äpfel schälen und in kleine Scheiben schneiden. Teig belegen und
bezuckern. Dann ca. 40 Minuten backen. Fertig, aus die Maus.
\end{quote}

\freeform\hrulefill\newline\freeform{Apfelkuchen à la Mama}\end{recipe}

\newpage

\begin{recipe}{Eingerührter Teig}{Für ca. 4 Personen}{10m}

\ingredient{1 dl}{Wasser}
\ingredient{100 g}{Butter\`}
\ingredient{1/2 EL}{Salz}

\begin{quote}
Zutaten in PFanne zergehen lassen.
\end{quote}

\freeform\hrulefill

\ingredient{250 g}{Mehl}

\begin{quote}
Mehl dazugeben und dann kühl stellen. Fertig.
\end{quote}

\freeform\hrulefill\newline\freeform{Rührteig für Allerlei}\end{recipe}

\newpage

\begin{recipe}{Mürbteig}{Für ca. 4 Personen}{10m}

\ingredient{70g}{Zucker}
\ingredient{140 g}{Butter}
\ingredient{210 g}{Mehl}
\ingredient{1}{Ei}
\ingredient{1 Pkg}{Vanillezucker}

\begin{quote}
alles zu einem Teig zusammenkneten und dann kaltstellen.
\end{quote}

\freeform\hrulefill\newline\freeform{Omas 123 Rezept}\end{recipe}

\newpage

\begin{recipe}{Quarktorte}{Für ca. 8 Personen}{2h}

\ingredient{70 g}{Zucker}
\ingredient{140 g}{Butter (weich)}
\ingredient{210 g}{Mehl}
\ingredient{1}{Vanillezuck}
\ingredient{1}{Ei}

\begin{quote}
Zusamenkneten und eine halbe Stunde kaltstellen. Dann damit den Boden
eine Springfrom bestreichen.
\end{quote}

\freeform\hrulefill

\ingredient{3}{Eigelb}
\ingredient{1 Tasse}{Zucker}

\begin{quote}
gut mixen.
\end{quote}

\freeform\hrulefill

\ingredient{3}{Eiweiss}

\begin{quote}
steifschlagen
\end{quote}

\freeform\hrulefill

\ingredient{750 g}{3/4-Fett Quark}
\ingredient{200 g}{Crème fraiche}
\ingredient{2 EL}{Mehl}

Vorsichtig mit Eigelbmasse und Eiweiss mischen und damit Springform
auffüllen. Bei 180° mit starker Unterhitze ca. 40' backen.

\freeform\hrulefill\newline\freeform{}\end{recipe}

\newpage

\begin{recipe}{Reisauflauf}{Für ca. 4 Personen}{1 h}

\ingredient{1 L}{Milch}
\ingredient{1}{Prise Salz}

\begin{quote}
Milch mit Salz zum kochen bringen.
\end{quote}

\freeform\hrulefill

\ingredient{200 g}{Milchreis}

\begin{quote}
Milchreis waschen und zur Milch dazugeben bis der Reis aufgequollen ist
und breiig wird. Dabei ständig rühren, aber nur langsam kochen lassen.
Danach. abkühlen lassen.
\end{quote}

\freeform\hrulefill

\ingredient{3-4}{Eier}
\ingredient{4 EL}{Zucker}
\ingredient{1 Päckchen}{Vanillezucker}

\begin{quote}
schaumig rühren.
\end{quote}

\freeform\hrulefill

\ingredient{100 g}{Rosinen}
\ingredient{2-4}{Äpfel}

\begin{quote}
Rosinen und kleingeschnittene Äpfel unter den Reis mischen. Die Masse in
eine gut gebutterte feuerfeste Form geben. Ofen vorheizen auf 180 Grad
ca. Ober und Unterhitze oder Umluft dann 200 Grad.
\end{quote}

\freeform\hrulefill

\ingredient{1-2 EL}{Zucker}
\ingredient{etwas}{Semmelbrösel}
\ingredient{1 EL}{Butter}

\begin{quote}
Butter in Flöckchen schneiden. Zucker, Semmelblödel und Butterschätzli
darüberstreuen. Ab in den Ofen für ca 30-40 Minuten.
\end{quote}

\freeform\hrulefill\newline\freeform{Omas Reisauflauf, ursprünglich von Uroma in Rothenburg.}\end{recipe}

\newpage

\begin{recipe}{Rüblitorte}{Für ca. 8 Personen}{1h}

\ingredient{4}{Eier}
\ingredient{300g}{Zucker}

\begin{quote}
Eigelb mit Zucker schaumig rühren; Eiweiss zu Schnee schlagen
\end{quote}

\freeform\hrulefill

\ingredient{300 g}{Karotten}
\ingredient{300 g}{Geriebene Mandeln}
\ingredient{1}{Zitrone}
\ingredient{1/2}{Päckchen Backpulver}
\ingredient{80 g}{Mehl}
\ingredient{2 EL}{Paniermehl}

\begin{quote}
Karotten raffeln und alles vorsichtig mit dem Eigelb und dann mit dem
Eiweiss vermischen. In eine gut ausgebutterte und am Boden mit
Backpapier belegte Springform geben. 180° ca. 50m.
\end{quote}

\freeform\hrulefill

\ingredient{5-6 EL}{Puderzucker}
\ingredient{1}{Zitrone}
\ingredient{6-12}{kleine Marzipanrübli}

\begin{quote}
Puderzucker mit Zitronensaft zu dickflüssiger Masse verarbeiten und auf
den ausgekühlten Kuchen giessen; mit Marzipanrübli dekorieren
\end{quote}

\freeform\hrulefill\newline\freeform{}\end{recipe}

\newpage

\begin{recipe}{Tiramisu}{Für ca. 4 Personen}{60m}

\ingredient{4}{Eier}

\begin{quote}
Eiweiss von Eigelb trennen. Eiweiss schlagen.
\end{quote}

\freeform\hrulefill

\ingredient{6 EL}{Zucker}

\begin{quote}
zusammen mit Eigelb schaumig schlagen. Eiweiss und Eigelbmasse
zusammenmischen.
\end{quote}

\freeform\hrulefill

\ingredient{ca. 1/2 Tafel}{halbbittere Schoggi}

\begin{quote}
Schoggi in kleine Brösel schneiden (mahlen)
\end{quote}

\freeform\hrulefill

\ingredient{1 Pckg.}{Mascarpone}

\begin{quote}
Mascarpone zur Masse dazugeben.
\end{quote}

\freeform\hrulefill

\ingredient{5}{Espressi (so ungefähr)}

\begin{quote}
Espressi machen. Schokoloade zur Masse geben. Etwas Milch in den Kaffee
tun. Einen Teil der Masse in eine niegrige grosse Form geben.
\end{quote}

\freeform\hrulefill

\ingredient{1 Pckg.}{Löffelbiskuit}

\ingredient{etwas}{Cognac}

\begin{quote}
Löffelbiskuit in Cognac wälzen und auf die Masse geben. Den Rest des
Cognacs auf die Löffelbisquit leeren. Eine weitere,jetzt recht dicke
Schicht Teig drauf. Dann wieder Schickt Biskuit (in Kaffee und Milch
wälzen). Dann auf Teig geben. Restlichen Teig drauf, weitere
Schoggisstreusel drauf und ab in den Kühlschrank.
\end{quote}

\freeform\hrulefill\newline\freeform{}\end{recipe}

\end{document}
